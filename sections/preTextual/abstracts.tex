% resumo em português
\setlength{\absparsep}{18pt} % ajusta o espaçamento dos parágrafos do resumo
\begin{resumo}
  \cp{
    O armazenamento é um dos componentes mais importantes da cadeia de 
    distribuição de vacinas, principalmente pela sensibilidade delas a variações 
    de temperatura, que podem ocasionar diminuição da sua eficácia. Considerando 
    isso, existe uma necessidade de manter constante monitoramento dessa variável, 
    afim de garantir que o produto final venha a manter suas característica 
    originais e a eficiência esperada. Esse trabalho apresenta uma solução baseada em Internet das Coisas, usando comunicação sem fio de baixa potência, para monitorar a qualidade de vacinas, por meio das medidas de temperatura e umidade nos locais de armazenamento. Os dados coletados são organizados em um banco de dados, podendo ser acessados por sistemas decisórios afim de avaliar a qualidade da vacina antes de sua aplicação, evitando problemas em decorrência de problemas de armazenamento.
  }

  \textbf{Palavras-chaves}: Monitoramento. Vacinas. IoT. LoRa.
\end{resumo}

% resumo em inglês
\begin{resumo}[Abstract]
  \begin{otherlanguage*}{english}
    \cp{
      Storage is one of the most important components of the vaccine distribution chain, mainly due to its sensitivity to temperature variations, which can cause a decrease in its effectiveness. Considering this, there is a need to keep constant monitoring of this variable, to guarantee that the final product will maintain its original characteristics and the expected efficiency. This work presents a solution based on the Internet of Things, using low power wireless communication, to monitor the quality of vaccines, by measuring temperature and humidity in storage locations. The collected data are organized in a database, which can be accessed by decision systems to assess the quality of the vaccine before its application, avoiding problems due to storage problems.
    }

    \noindent
    \textbf{Key-words}: Monitoring. Vacaciones. IoT. LoRa.
  \end{otherlanguage*}
\end{resumo}
