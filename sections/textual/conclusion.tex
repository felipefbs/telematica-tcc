\chapter{Considerações Finais}
\label{cap:conclusao}

O presente trabalho apresentou um sistema de monitoração de produtos imunobiológicos, visando o auxílio dos profissionais da saúde em manter a qualidade dos materiais. Tal solução é composta por um aplicativo móvel para o cliente realizar a monitoração, um servidor em nuvem, responsável por concentrar os dados coletados e dos usuários e dos hardwares, que captam as temperaturas e humidades e transmitem para o servidor.

Foi comprovado na sessão \ref{result:transmissao} a efetividade dos hardwares de transmitir os dados coletados em uma distância considerável, cerca de 60 metros, em um ambiente com inúmeros obstáculos causando bastante interferência, proporcionado uma taxa de entrega de pacotes de 78,82\%, um valor alto para o ambiente proposto.
 
Em relação ao custo total de confecção do produto, comprado realizada localmente, ficou  em torno de R\$ 180,00 para o \textit{gateway} e R\$ 97,66 para cada \textit{end node}, a quantidade de end nodes varia conforme a quantidade de ambientes de armazenagem do usuário, uma para cada câmera de conservação de termolábeis. Em relação ao \textit{gateway}, será necessário adquirir outra unidade apenas em caso de algum \textit{end node} ficar localizado em uma distância grande ao ponto do receptor não conseguir receber os pacotes, já que que nas condiçòes estabelecidas, utilizando o LoRaWAN, cada \textit{gateway} pode aguentar, teoricamente, cerca de 62 mil \textit{end node} \cite{lora2021specification}. Considerando a possibilidade de realizar a comprar dos componentes importando da China, o preço dos produtos pode chegar a 50\% a menos em comparação aos preços locais, além de, pensando na produção dos dispositivos em massa, esse preço tende a cair, entretanto, não foi realizado um estudo para saber o quanto de economia teria.

\todo {adicionar um parágrafo falando do consumo energético}

% ---
\section{Sugestões para Trabalhos Futuros}
\label{conclusao:futuros}
Há diversas melhorias que podem ser feitas neste projeto. O LoRa possui um ajuste da potência de transmissão que varia entre +7 dBm a +20 dBm, quanto maior a potência, maior o alcance de transmissão, entretanto, o consumo também cresce. É possível realizar um teste entre esses valores em busca de um melhor balanço entre economia energética e alcance de transmissão, seria interessante também adicionar uma configuração no dispositivo onde o usuário possa escolher qual potência usar, conforme for a sua demanda.

Outra sugestão seria adicionar um nova funcionalidade ao \textit{end nodes} para detectarem que estão em em um ambiente fora do recomendado, ou se teve uma variação considerável em um curto período de tempo da temperatura do ambiente, para que possam emitir um alerta, onde o aplicativo recebe, em tempo real, tal informação.

Olhando para o consumo energético, apesar de termos alcançado um bom resultado, ainda é possível diminuir tal consumo utilizando um \textit{clock} menor, o componente responsável por isso é o cristal oscilador, que neste projeto fui utilizando um de 16 mHz,  entretanto, esta aplicação não necessita dessa \textit{clock} alto, podendo assim utilizar um cristal oscilador de valor menor, inclusive, o próprio Atmega328, possui um clock interno de 1 mHz, que pode ser utilizado no lugar do cristal oscilador, o que traz dois benefícios, uma diminuição dos componentes para o circuito e uma economia de energia.

Um teste interessante a ser realizado, e uma análise em relação a quantidade de dispositivos conectados simultaneamente a um \textit{gateway}, entretanto, necessitaria de muitos dispositivos para realizar tal teste, o que dificultaria na sua realização, considerando que, teoricamente, um \textit{gateway} suporta 1.5 milhões de pacotes por dia, e cada \textit{end node} transmite um pacote por hora, totalizando 24 pacotes por dia.

Este trabalho teve um foco maior na construção do \textit{end node}, o deixou o \textit{gateway} com um grande margem de melhoria, como a construção de um protótipo com apenas as peças necessárias para seu funcionamento, que resultaria em um custo benefício maior e uma economia energética melhor.

Por fim, o servidor foi construído de uma forma modular, que facilita a expansão de novas funcionalidades em trabalhos futuros e pensando na usabilidade do sistema como um todo, tentaria melhorar a usabilidade do aplicativo móvel, em torno do gerenciamento dos dispositivos cadastrados, na atribuição dos identificados aos equipamentos, entre outros.