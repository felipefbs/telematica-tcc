\chapter[Introdução]{Introdução}
\label{cap:intro}
A saúde é um fator de suma importância para todos os seres vivos, ele é um problema científico, tecnológico, político, prático e filosófico que refere-se a um estado completo de bem estar físico, emocional, social, intelectual e espiritual \cite{almeida2011saude}. 

Segundo o artigo 196 \cite{de2013direito} da Constituição Federal Brasileira a saúde é um direito de todos e dever do Estado garantir medidas políticas sociais e econômicas que visam à diminuição do risco de doenças e de outros agravamentos e ao acesso universal e imparcial às ações e serviços para a sua promoção, proteção e recuperação.

Para garantirmos nossa saúde, precisamos cuidar do nosso corpo e mente. Para isto, uma ferramenta que podemos contar são os imunobiológicos, como as vacinas e os soros. Diferente de remédios que ajudam no tratamento de pessoas doentes, as imunobiológicos são uma preparação biológica que fornece imunidade total ou parcial de uma determinada doença autoimune para um indivíduo saudável. As vacinas e os soros se diferem pela sua forma de imunização, as vacinas fornecem uma imunização ativa, estimulando o nosso organismo na produção de anticorpos, os soros fornecem uma imunização passiva, provendo os anticorpos para o nosso organismo que foram produzidos  em outros organismo \cite{soma2018tratamento}.

Contudo, os imunobiológicos requerem um cuidado elevado para manter a qualidade e sua eficiência, um dos fatores é que são produtos termolábeis, ou seja, se deterioram após determinado tempo expostos a variações de temperaturas e umidade, portanto, é imprescindível assegurar que seu ambiente de armazenagem mantenha uma temperatura e umidade constante \cite{ministerio2001manual} para garantir uma longevidade maior para o produto. Para este propósito, existe a Rede de Frio, um processo desenvolvido pelo Programa Nacional de Imunizações, PNI, de conservação, armazenamento e transporte dos medicamentos, objetivando as condições adequadas dos mesmos, mantendo suas características iniciais \cite{ministerio2001manual}.

No ano de 2014, foi relatado no estudo \cite{oliveira2014avaliaccao} que a qualidade de conservação das vacinas não eram adequadas em boa parte dos municípios da macrorregião Oeste de Minas Gerais, alguns dos motivos citados foram a má gestão dos refrigeradores, falhas no monitoramento da temperatura e insuficiência de recursos humanos.

% ---
\section{O Programa Nacional de Imunizações}
\label{intro:PNI}
Com o sucesso da Campanha de Erradicação da Varíola, CEV, iniciada em 1965, tendo seu fim em 1973 \cite{muniz2011memorias}, amplificou dentro do Ministério da Saúde maiores investimentos no controle de doenças autoimune, dando um impulso na criação do PNI \cite{temporao2003programa}. O PNI foi fundado com objetivo de controlar e erradicar as doenças imunopreveníveis, através de ações metalizadas de vacinação da população. Em 1980 foi realizada a primeira campanha de vacinação da poliomielite e desde então foram realizadas diversas campanhas, tais como a da rubéola, sarampo, tuberculose, febre amarela \cite{temporao2003programa, ministerio2001manual} e atualmente contra a COVID-19.

De acordo com a Lei n.º 6.259 de 30 de outubro de 1975, regularizada pelo Decreto nº 78.231 em 1976, certificar o PNI, sobre a responsabilidade do Ministério da Saúde e define as seguintes competências \cite{ministerio2001manual}:

  \begin{itemize}
    \item implantar e implementar as ações do Programa, relacionadas com as vacinações de caráter obrigatório;
    \item  estabelecer critérios e prestar apoio técnico e financeiro à elaboração, implantação e implementação dos programas de vacinação a cargo das secretarias de saúde das unidades federadas;
    \item estabelecer normas básicas para a execução das vacinações;
    \item supervisionar, controlar e avaliar a execução das vacinações no território nacional, principalmente o desempenho dos órgãos das Secretarias de Saúde, encarregados dos programas de vacinação.
  \end{itemize}

% ---
\section{Rede de Frio}
\label{intro:redes-de-frio}
A Rede de Frio, também chamada de Cadeia de Frio é um processo definido pelo PNI designado a auxiliar os profissionais da área da saúde, responsáveis pela imunização no Brasil, para que possa assim, garantir a efetividade e durabilidade dos imunobiológicos e medicamentos termolábeis.

No Manual de Rede de Frio \cite{ministerio2001manual} são definidos os requisitos dos ambientes de armazenagem para garantir a efetividade dos produtos, desde os laboratórios produtores às instâncias locais, passando pela instância nacional, estadual, e no transporte entre eles. Para as  Câmaras frigoríficas, a temperatura de operação é entre -20°C a +2°C, variando conforme o material armazenado. Para a maioria dos imunobiológicos o recomendado é de +2°C e +8°C para ter um melhor controle da sua validade, havendo algumas exceções, como por exemplo, as vacinas Pfizer-BioNTech e Moderna,  produzidas para combater o COVID-19, que precisam ser armazenadas entre –80°C a –60°C e –25°C e –15°C, respectivamente \cite{niforatoscommon}.

% ---
\section{Justificativa e Relevância do Trabalho}
\label{intro:justificativa}
Atualmente, com o início da distribuição das vacinas contra o COVID-19 em todo o mundo, uma das dificuldades enfrentadas é o controle de qualidade no armazenamento tanto em transporte \cite{baechallenges}, quanto no local da aplicação justamente por serem produtos sensíveis à temperatura e necessitarem de muita cautela. Em contrapartida, por ser um produto com uma demanda elevada, a sua oferta deve ser rápida para que haja imunização em massa da população, encaminhando-se para o fim da pandemia.

Esses desafios enfrentados na distribuição das vacinas do COVID-19 não são as únicas dificuldades para a garantia da qualidade dos imunobiológicos. Enfermeiras de Minas Gerais, com o objetivo de inteirar-se acerca do sistema de manutenção dos produtos, realizaram um estudo sobre a conservação de vacinas em unidades básicas de saúde, UBSs. Nessa pesquisa foram relatadas diversas irregularidades no armazenamento dos materiais termolábeis sem o comprimento das normas da PNI, como por exemplo, a presença de vacinas que deveriam ter sido descartadas por terem atingido seu tempo máximo de diluição, ainda presentes nos refrigeradores. Cerca de 52\% dos imunobiológicos armazenados nos refrigeradores eram acomodados erroneamente e 36\% dos refrigeradores observados contavam com objetos em portas, como fracos vazios.

Analisando as temperaturas dos ambientes de armazenagem, foi observado que 4\% das unidades não realizavam o registro da temperatura dos refrigeradores, 88\% dos refrigeradores usavam termômetros analógicos de baixa confiabilidade e cerca de 12\% estavam com temperatura abaixo da faixa recomendada, chegando a 0°C. Outros estudos realizado \cite{oliveira2014avaliaccao, nelson2007monitoring, falcon2020vaccine} também testemunharam essa irregularidade na temperatura em seus respectivos locais ao redor do mundo, valendo salientar o estudo realizado na Bolívia \cite{nelson2007monitoring}, que teve resultados ainda piores, sendo registado uma temperatura mínima de -7.2°C e uma máxima de 22.7°C.

Pensando nesse cenário, esse trabalho apresenta uma alternativa para melhorar a forma de monitoramento de temperatura e umidade realizada em produtos imunobiológicos por laboratórios, unidades públicas de saúde e afins, no intuito de auxiliá-los a manter os materiais em suas melhores condições.

% ---
\section{Objetivos}
\label{intro:objetivos}

\subsection{Objetivo Geral}
\label{intro:objetivos:geral}
Construir uma solução baseada em conceitos de IoT visando o monitoramento de temperatura e umidade de imunobiológicos para auxiliar funcionários da saúde, garantindo melhores condições para a vacinação da população frente a incidência de doenças.

\subsection{Objetivos Específicos}
\label{intro:objetivos:especificos}
\begin{itemize}
  \item Construir um protótipo inicial para coleta da temperatura e umidade nos ambientes de armazenagens dos imunobiológicos.
  \item Implementar um servidor para a armazenagem dos dados coletados e posteriormente fornecer históricos das temperaturas e umidade ao aplicativo móvel.
  \item Desenvolver um aplicativo móvel para fornecer uma interface amigável para os usuários, auxiliando no controle de qualidade dos produtos.
  \item Realizar testes e análises dos dados transmitidos a fim de garantir a confiabilidade das temperaturas e umidade coletadas.
\end{itemize}

% ---
\section{Trabalhos Relacionados}
\label{fund:trabalhos-relacionados}
Estudantes do Instituto Federal da Bahia desenvolveram uma pesquisa \cite{cruzdesenvolvimento} sobre diferentes métodos de construção de um sensor de temperatura de baixo custo voltado para o monitoramento de vacinas, pois a temperatura é uma das variantes de grande importância no  processo de conservação de materiais imunobiológicos.

No estudo \cite{lima2019controle}, foi desenvolvido um produto de monitoramento de temperaturas de caixas térmicas usadas no transporte de materiais termolábeis, com o  objetivo de alertar os transportadores de quando a condição de armazenagem estiver fora da faixa ideal, disposdo de sinais auditivos e visíveis. No seu protótipo foi utilizado uma tela LCD para visualizar a temperatura atual, um led vermelho e outro verde para informar visivelmente sobre a condição do ambiente, um sensor de temperatura DS18B20, para coletar os dados necessários e um Arduíno Uno R3, como controlador dos equipamentos.

Em \cite{kersbaum2019monitoramento}, um estudo similar ao citado posteriormente, foi construído um dispositivo de baixo custo para o monitoramento de temperaturas voltado para geladeiras domésticas que armazenam vacinas em municípios do Brasil de pequeno porte, pois possuem baixa quantidade de insumo que consequentemente, seus ambientes de armazenagem de materiais termolábeis são precários, possuindo poucas câmeras específicas para armazenagem deste tipo de material, que acabam utilizando geladeiras domésticas. Seu protótipo é composto por um Arduino conectado a uma célula de Peltier, refrigerado por ventoinhas acoplado em uma caixa de isopor, os dados da temperatura são coletados pelo Arduino e enviado para um computador onde pode ser visualizado por um web site e armazenado em uma planilha excel.

Na dissertação de Mestrado de Lopes Neto \cite{lopes2019monitoramento}, foi desenvolvido uma plataforma em Arduino para transmissão de dados de temperatura do interior dos ambientes gerais de conservação de materiais sensíveis à temperatura, transmitido via SMS e GPRS para uma central onde é armazenado os dados para consultas posteriores.

Tendo em vista as pesquisas mencionadas acima, venho através deste trabalho, contribuir para uma solução mais completa, do hardware ao cliente final, de baixo custo e energeticamente eficaz, utilizando tecnologias modernas e escaláveis, com o intuito de auxiliar os profissionais da saúde no controle dos imunobiológicos.

% ---
% \section{Metodologia}
% \label{intro:metodologia}

% No intuito de alcançar os objetivos pretendidos, a metodologia utilizada neste trabalho foi composta pelas seguintes etapas:

% \begin{itemize}
%   \item \todo{Adicionar os pontos}.
% \end{itemize}

% ---
% \section{Estrutura do Documento}
% \label{intro:estrutura}

% Os capítulos subsequentes estão organizados da seguinte maneira:

% \hm{A ser feito quando o documento tiver pronto}