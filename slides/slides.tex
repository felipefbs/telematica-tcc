%%%%%%%%%%%%%%%%%%%%%%%%%%%%%%%%%%%%%%%%%%%%%%%%%%%%%%%%%%%%%%%%%%%%
%% I, the copyright holder of this work, release this work into the
%% public domain. This applies worldwide. In some countries this may
%% not be legally possible; if so: I grant anyone the right to use
%% this work for any purpose, without any conditions, unless such
%% conditions are required by law.
%%%%%%%%%%%%%%%%%%%%%%%%%%%%%%%%%%%%%%%%%%%%%%%%%%%%%%%%%%%%%%%%%%%%

% This theme was based on fibeamer theme 
% If you found any bugs please contact @karlosos
% This repository is hosted on github https://github.com/karlosos/zut-fibeamer/

\documentclass[t]{beamer}
% \documentclass{beamer}
\usetheme[faculty=wi]{fibeamer}
\usepackage[utf8]{inputenc}
\usepackage[
  main=portuguese,
  english
]{babel}

\title{Sistema de Monitoramento de Qualidade de Imunobiológicos na Cadeia de Distribuição e Armazenamento}
\subtitle{Henrique M. Miranda}
\author{Orientador: Paulo Ribeiro Lins Junior}

\usepackage{ragged2e}  % `\justifying` text
\usepackage{booktabs}  % Tables
\usepackage{tabularx}
\usepackage{tikz}      % Diagrams
\usetikzlibrary{calc, shapes, backgrounds}
\usepackage{amsmath, amssymb}
\usepackage{url}       % `\url`s
\usepackage{listings}  % Code listings
\frenchspacing
\begin{document}
\frame[c]{\maketitle}

% \AtBeginSection[]{% Print an outline at the beginning of sections
%   \begin{frame}<beamer>
%     \frametitle{Seção \thesection}
%     \tableofcontents[currentsection]
%   \end{frame}
% }

\begin{darkframes}
  % \section{Introdução}
  % \begin{frame}{Introdução}

  % \end{frame}

  \section{Objetivos}
\begin{frame}{Objetivos}
  \framesubtitle{\textcolor{fibeamer@black}{Objetivo Geral}}
  
  Construir uma solução baseada em conceitos de IoT visando o monitoramento de temperatura e umidade de imunobiológicos para auxiliar funcionários da saúde, garantindo melhores condições para a vacinação da população frente a incidência de doenças.
\end{frame}

% \begin{frame}{Objetivos}
%   \framesubtitle{Objetivos Específico}
%   \begin{itemize}
%     \item {Construir um protótipo inicial para coleta da temperatura e umidade nos ambientes de armazenagens dos imunobiológicos.}
%     \item {Implementar um servidor para a armazenagem dos dados coletados e posteriormente fornecer históricos das temperaturas e umidade ao aplicativo móvel.}
%     \item {Desenvolver um aplicativo móvel para fornecer uma interface amigável para os usuários, auxiliando no controle de qualidade dos produtos.}
%     \item {Realizar testes e análises dos dados transmitidos a fim de garantir a confiabilidade das temperaturas e umidade coletadas.}
%   \end{itemize}
% \end{frame}
  \chapter{Resultados}
\label{cap:result}

% ---
\section{Análise de transmissão dos pacotes}
\label{result:transmissao}

% ---
\section{Análise do consumo de bateria}
\label{result:consumo}
\end{darkframes}
\end{document}
