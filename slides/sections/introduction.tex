\section{Introdução}
\begin{frame}{Introdução}
  \framesubtitle{Vacinas contra COVID-19}
  
  \begin{itemize}
    \item Possuem uma demanda elevada
    \item Produto sensível à temperatura
    \item Controle de qualidadde
    \item Armazenagem e transporte
  \end{itemize}
\end{frame}

\begin{frame}{Introdução}
  \framesubtitle{Recomendações da PNI}
  
  \begin{itemize}
    \item Manual de Rede de Frio
    \item Umidade mínima de \alert{5\%}
    \item Armazenamento entre \alert{2°C} e \alert{8°C}
    \item Exceções: Pfizer-BioNTech e Moderna
    \item Registrar pelo menos 2 vezes por dia
  \end{itemize}
\end{frame}

\begin{frame}{Introdução}
  \framesubtitle{Estudo em Minas Gerais}
  
  \begin{itemize}
    \item Estudo sobre conservação de vacinas em UBSs
    \item Objetivo: inteirar-se acerca do sistema de manutenção dos produtos
    \item Relatadas diversas irregularidades
    \begin{itemize}
      \item Vacinas vencidas ainda presentes nos refrigeradores
      \item \alert{4\%} das unidades não realizavam o registro de temperatura
      \item \alert{88\%} dos refrigeradores usavam termômetros analógicos
      \item \alert{14\%} dos refrigeradores estavam com temperatura abaixo da faixa recomendada
    \end{itemize}
  \end{itemize}
\end{frame}
